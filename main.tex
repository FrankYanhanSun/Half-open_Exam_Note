\documentclass[UTF8, fontset=none]{ctexart}
%\usepackage[heading=true]{ctex}

\usepackage{xeCJK}
\usepackage{fontspec}
\setCJKmainfont{Noto Serif CJK SC} % 设置中文字体为 Noto Serif CJK SC

\usepackage{amsmath, amssymb}

\usepackage[a4paper, 
            total={6in, 8in},  % 精确控制内容区域
            left=2mm, right=2mm,   % 左右边距压至mm
            top=2mm, bottom=2mm,   % 上边距3mm,下边距2mm
            landscape]{geometry}
\usepackage{multicol}
\setlength{\columnsep}{1mm} % 栏间距进一步缩小
\usepackage{titlesec}
\usepackage{setspace}
\usepackage{lipsum}


% 设置极紧凑行距(0.7倍行距)
\setstretch{0.7}
\setlength{\parskip}{0pt}
\setlength{\parindent}{0em} % 缩进减小

% 数学公式字号适配(8pt 下公式需同步缩小)
\DeclareMathSizes{8}{7}{6}{5} % 正文公式/行内公式/上下标字号

% 四级标题格式(字号逐级缩小,间距压至极限)
\titleformat{\section}{\normalfont\footnotesize\bfseries}{\thesection}{0.5em}{}
\titlespacing{\section}{0pt}{0.5ex}{0.5ex}

\titleformat{\subsection}{\normalfont\scriptsize}{\thesubsection}{0.5em}{}
\titlespacing{\subsection}{0pt}{0.4ex}{0.4ex}

\titleformat{\subsubsection}{\normalfont\tiny\itshape}{\thesubsubsection}{0.5em}{}
\titlespacing{\subsubsection}{0pt}{0.3ex}{0.3ex}

\titleformat{\paragraph}{\normalfont\tiny\bfseries}{\theparagraph}{0.5em}{}
\titlespacing{\paragraph}{0pt}{0.2ex}{0.2ex}

\begin{document}
\begin{multicols*}{8}
\zihao{8} % 设置中文字体大小为 \zihao{8}(约 6pt)
% 内容区域
\section{半导体}
\subsection{杂质半导体}
本征半导体[4价,绝缘体,本征激发(温升)形成空穴和电子,都是载流子]   =>掺杂\\
杂质半导体\\
N型:多子(电子),少子(空穴),掺5价\\
P型:多子(空穴),少子(电子),掺3价\\
少子来源于本征激发\\

\subsection{PN结:单向导通}
没有外加电场:多子扩散与少子漂移动态平衡,没有电流\\
*空间电荷电荷区,无载流子(耗尽层)
内电场阻碍多子扩散,外电场与内电场相反时导通

\subsection{二极管}
\subsubsection{伏安特性}%伏安特性曲线+近似
*\\
*\\
*\\
*\\
*\\
*\\
*\\
*\\
*\\
*\\
*\\
*\\
*
\subsubsection{电路分析}
将电路内二极管全部看为断路,分析两端电势差是否能使二极管导通
(压降有时可忽略)\\
*当无电流时(断路),电阻不会产生压降\\

\subsubsection{稳压二极管:工作在反向击穿区}
电流变化大,电压变化小\\%工作点+例子
看作一个恒压源\\
*\\
*\\
*\\
*\\
*\\
*\\
*\\
*\\
*\\
*\\
*\\
*
\subsection{三极管}
\subsubsection{原理}
掺杂浓度:发射级E>集电极C>基级E\\%结构图+电路图
*\\
*\\
*\\
*\\
*\\
*\\
*\\
*\\
*\\
*\\
*\\
*\\
*
\[I_E =I_C (+ I_B)\]
\[I_C = \beta I_B\]
发射极:正向偏置\\
集电极:负向偏置\\
*\\
*\\
*\\
*\\
*\\
*\\
*\\
*\\
*\\
*\\
*\\
*
\subsubsection{判断}
B、E间电势判断材料,C极远大(小)于判断NPN(PNP)\\%例子
*\\
*\\
*\\
*\\
*\\
*\\
*\\
*\\
*\\
*\\
*\\
*\\
*\\
*\\
*
\subsubsection{特征曲线}%输入输出特征曲线+三种工作状态
*\\
*\\
*\\
*\\
*\\
*\\
*\\
*\\
*\\
*\\
*\\
*\\
*\\
*\\
*


\section{基本放大电路}
\subsection{固定式共发射极}%单电源电路图
*\\
*\\
*\\
*\\
*\\
*\\
*\\
*\\
*\\
*\\
*\\
*
\subsubsection{静态分析:估算法求静态工作点}%静态电路图+静态工作点Q
*\\
*\\
*\\
*\\
*\\
*\\
*\\
*\\
*\\
*
\[U_cc - I_B R_B - U_BE = 0\]
\[U_cc - I_C R_C - U_{CE} = 0, I_C = \beta I_B\]
\subsubsection{动态分析}
\paragraph{微变等效电路}%微变等效电路
*\\
*\\
*\\
*\\
*\\
*\\
*\\
*\\
*\\
*
\paragraph{静态工作点Q}
电压放大倍数$A_u$\\
输入电阻$r_i$\\
输出电阻$r_0$
\paragraph{输入电阻}
\[r_i = r_{be} = 200\Omega + (1 + \beta)\frac{26 mV}{I_E mA}\]
$I_E$为静态下发射极电流
\paragraph{输出电阻}
\[r_0 = R_{C}\]
\paragraph{电压放大倍数}
\[A_u = -\beta \frac{R_C//R_L}{r_{be}}\]
*注意后面负载的并联情况\\
\subsubsection{特性}
输入输出相反\\
放大倍率高\\
输入电阻低(越大越好)\\
输出电阻高(越小越好)
\subsubsection{非线性失真}%Q失真
*\\
*\\
*\\
*\\
*\\
*\\
*\\
*\\
*\\
*\\
*
\subsection{分压式共发射极}
可自动调节$I_B$使Q稳定\\%电路图
*\\
*\\
*\\
*\\
*\\
*\\
*\\
*\\
*\\
*\\
*
\subsubsection{静态分析}
*\\
*\\
*\\
*\\
*\\
*\\
*\\
*\\
*\\
*\\
*\\
\[V_B = \frac{R_{b2}}{R_{b1} + R_{b2}} U_{cc}\]
\[I_C = I_E = \frac{V_B (- U_{BE})}{R_e}\]
\[I_B = \frac{I_C}{\beta}\]
\[U_{CE} = U_{cc} - I_C R_C - I_E R_E\]
\subsubsection{动态分析}
\paragraph{微变等效电路}
*\\
*\\
*\\
*\\
*\\
*\\
*\\
*\\
*\\
*\\
*
\paragraph{输入电阻}
\[r_{be} = 200\Omega + (1 + \beta)\frac{26 mV}{I_E mA}\]
\[r_i = r_{be} +(1 + \beta)R_e\]
$I_E$为静态下发射极电流
\paragraph{输出电阻}
\[r_0 = R_{C}\]
\paragraph{电压放大倍数}
\[A_u = -\beta \frac{R_C//R_L}{r_{be} + (1 + \beta)R_e}\]
相比固定式,$A_u$变小了
\subsection{含旁路电容分压式共发射极}
使Q稳定且$A_u$较大
\subsubsection{静态分析}
同分压式共发射极,旁路电容$C_e$断路
\subsubsection{动态分析}
同固定式共发射极,旁路电容$C_e$短路\\
*但需注意$R_{b1}$和$R_{b2}$,稍有不同\\
\paragraph{微变等效电路}
*\\
*\\
*\\
*\\
*\\
*\\
*\\
*\\
*\\
*\\
*
\paragraph{输入电阻}
\[r_i = r_{be} = 200\Omega + (1 + \beta)\frac{26 mV}{I_E mA}\]
$I_E$为静态下发射极电流
\paragraph{输出电阻}
\[r_0 = R_{C}\]
\paragraph{电压放大倍数}
\[A_u = -\beta \frac{R_C//R_L}{r_{be}}\]
\subsection{射级输出器}
电压无放大作用,电流有放大作用\\
输入电阻大,输出电阻小\\
放大倍率<=1
\subsection{多级耦合}
有零点漂移问题,需使用差分放大电路放在第一级抑制\\%再看
\subsection{差分放大电路}%不知道考不考

\section{集成运算放大器}
\subsection{特点}%引脚+电压输出特性
*\\
*\\
*\\
*\\
*\\
*\\
*\\
*\\
*\\
*\\
*\\
*\\
*\\
*
\paragraph{理想化}
\paragraph{线性区工作特点}
虚短:$u_+ = u_-$\\
虚断:$I_+ = I_- = 0$\\
\paragraph{饱和区工作特点}
电压:\\
$u_+ > u_-, u_0 = +U_{0(sat)}$\\
$u_+ < u_-, u_0 = -U_{0(sat)}$\\
电流:$I_+ = I_- = 0$\\
\subsection{运算:虚短虚断秒杀}
\subsubsection{比例}
\paragraph{反向比例}
*\\
*\\
*\\
*\\
*\\
*\\
*\\
*\\
*\\
*\\
*\\
\[u_0 = -\frac{R_F}{R_1}u_i\]
\paragraph{同向比例}
*\\
*\\
*\\
*\\
*\\
*\\
*\\
*\\
*\\
*\\
*\\
\[u_0 = (1 +\frac{R_F}{R_1})u_i\]
\paragraph{电压跟随器:相等或相反}
*注意输入输出反馈的正负\\
*\\
*\\
*\\
*\\
*\\
*\\
*\\
*\\
*\\
*\\
*
\subsubsection{加减法}
\paragraph{反向加法}%手写公式推导
*\\
*\\
*\\
*\\
*\\
*\\
*\\
*\\
*\\
*\\
*
\paragraph{同向加法}%手写公式推导
*\\
*\\
*\\
*\\
*\\
*\\
*\\
*\\
*\\
*\\
*
\paragraph{减法}%手写公式推导
*\\
*\\
*\\
*\\
*\\
*\\
*\\
*\\
*\\
*\\
*
\subsubsection{积分}
*流经电容电流:$i_C = C_F\frac{dU_C}{dt}$
*\\
*\\
*\\
*\\
*\\
*\\
*\\
*\\
*
\subsubsection{微分}
*流经电容电流:$i_C = C_F\frac{dU_C}{dt}$
*\\
*\\
*\\
*\\
*\\
*\\
*\\
*\\
*
\subsection{信号处理:电压比较器}
运放工作在饱和区\\%运放工作波形+电路+波形
*\\
*\\
*\\
*\\
*\\
*\\
*\\
*\\
*\\
\[U_i > U_i, U_0 = -U_0\]
\[U_i < U_i, U_0 = U_0\]
*\\
*\\
*\\
*\\
*\\
*\\
*\\
*\\
应用:正弦波转方波\\
*\\
*\\
*\\
*\\
*\\
*\\
*\\
*\\

\section{反馈}
\subsection{反馈概念}
输出对输入影响,促进即正反馈,抑制即负反馈\\%图
*\\
*\\
*\\
*\\
*\\
*\\
*\\
*\\
*\\
*\\
*
\subsection{交流反馈判断}
\subsubsection{串/并联}
并联:输入与反馈在同一端\\
串联:输入与反馈在不同端\\
\subsubsection{电压/电流}
电压:反馈从输出端直接引出\\
电流:反馈从负载电阻近地端\\
\subsubsection{正/负反馈:瞬时极性法}
设输入端为正,根据运放上的正负号推出输出端(即反馈端)极性,与反馈的作用点极性比较\\
*\\
串联反馈:输入与反馈\textbf{同极性}为\textbf{负反馈};\textbf{反极性}为\textbf{正反馈}\\
并联反馈:输入与反馈\textbf{同极性}为\textbf{正反馈};\textbf{反极性}为\textbf{负反馈}\\
\subsubsection{正/负反馈:相对极性法}
\subsection{四种负反馈电路}
\subsubsection{串联电压负反馈}
*\\
*\\
*\\
*\\
*\\
*\\
*\\
*\\
*\\
*\\
*\\
*\\
*
\subsubsection{并联电压负反馈}
*\\
*\\
*\\
*\\
*\\
*\\
*\\
*\\
*\\
*\\
*\\
*\\
*\\
*\\
*
\subsubsection{串联电流负反馈}
*\\
*\\
*\\
*\\
*\\
*\\
*\\
*\\
*\\
*\\
*\\
*\\
*\\
*\\
*
\subsubsection{并联电流负反馈}
*\\
*\\
*\\
*\\
*\\
*\\
*\\
*\\
*\\
*\\
*\\
*
\subsection{负反馈对放大电路性能的影响}
\subsubsection{放大倍数}
开环放大倍数$A = \frac{{{{\dot X}_0}}}{{{{\dot X}_d}}}$\\
反馈系数$F = \frac{{{{\dot X}_f}}}{{{{\dot X}_0}}}$\\
闭环放大倍数${A_f} = \frac{{{{\dot X}_0}}}{{{{\dot X}_i}}} = \frac{{{{\dot X}_0}}}{{{{\dot X}_d} + {{\dot X}_f}}} = \frac{A}{{1 + AF}} < A$\\
放大倍数下降,稳定性提升\\
*|AF| > 1,为深度负反馈\\%不考
\subsubsection{减小失真}%再看
\subsubsection{输入电阻和输入电阻}
\paragraph{输入电阻$r_i$}
串联反馈:$r_i$增大\\
并联反馈:$r_i$减小\\
\paragraph{输出电阻$r_0$}
电流反馈:$r_0$增大,可视为恒流\\
电压反馈:$r_0$减小,可视为恒压\\
\subsection{负反馈对放大电路性能的影响}
\subsubsection{自激振荡}
\paragraph{条件}
正反馈\\
$A_fF = 1 \rightarrow$\\
$|A_fF| = 1, \varphi_A + \varphi_F = 2n\pi $\\
*\\
*\\
*\\
*\\
*\\
*\\
*\\
*\\
*\\
*\\
*\\
*\\
*

\section{直流稳压电源}
交流 -> 直流\\
变压(降压) -> 整流(交转直) -> 滤波 -> 稳压\\%流程图
*\\
*\\
*\\
*\\
*\\
*\\
*\\
\subsection{整流电路}
\subsubsection{单向半波整流}%电路+电压
*\\
*\\
*\\
*\\
*\\
*\\
*\\
*\\
*\\
*\\
*\\
*\\
*\\
电压平均值:${U_0} = \frac{1}{{2\pi }}\int_0^\pi  {\sqrt 2 U\sin \omega td(\omega t) = 0.45U}$\\
电流平均值:${I_0} = \frac{{{U_0}}}{{{R_L}}} = 0.45\frac{U}{{{R_L}}}$\\
每管电流平均值:$I_D = I_0$\\
故障分析:\\
D短路:无整流,$u_D = 0, u_0 = u$\\
D断路:无输出,$u_D = u, u_0 = 0$\\

\subsubsection{单向桥式整流}%电路+电压
*\\
*\\
*\\
*\\
*\\
*\\
*\\
*\\
*\\
*\\
*\\
*\\
*\\
*\\
*\\
电压平均值:\\
${U_0} = \frac{2}{{2\pi }}\int_0^\pi  {\sqrt 2 U\sin \omega td(\omega t) = 0.9U} $\\
电流平均值:${I_0} = \frac{{{U_0}}}{{{R_L}}} = 0.9\frac{U}{{{R_L}}}$\\
每管电流平均值:$I_D = \frac{1}{2}I_0$\\
故障分析:\\
$D_2$或$D_4$断路,即半波整流\\
$D_2$或$D_4$短路,电源短路,烧坏\\
$D_2$或$D_4$接反,电源短路,烧坏\\

\subsection{滤波电路}
\subsubsection{电容}
原理:$u_0 = u_C$\\
$u > u_C$时,电源给电容充电\\
$u < u_C$时,电容给负载放电,$u_C = u_0$下降慢\\
*\\
*\\
*\\
*\\
*\\
*\\
*\\
*\\
*\\
*\\
*\\
*\\
*\\
*\\
*\\
*\\
*\\
*\\
桥式电压:$U_0 = 1.2U$\\
半波电压:$U_0 = 1.0U$\\
\subsubsection{电感电容}
适合大电流、高频、纹波要求小的场景\\
原理:电感L隔交流通直流\\
*\\
*\\
*\\
*\\
*\\
*\\
*\\
*\\
*\\
*\\
*\\
*\\
*
\subsubsection{$\pi $型:LC、RC}
*\\
*\\
*\\
*\\
*\\
*\\
*\\
*\\
*\\
*\\
*\\
*\\
*
\subsection{稳压电路}
\subsubsection{稳压二极管}
*\\
*\\
*\\
*\\
*\\
*\\
*\\
*\\
*\\
*\\
*\\
*\\
*
\subsubsection{串联型电压负反馈}
四个部分:调整元件、比较放大、基准电压、取样电路\\
\textbf{串联电压负反馈}\\
*\\
*\\
*\\
*\\
*\\
*\\
*\\
*\\
*\\
*\\
*\\
*\\
*

\end{multicols*}
\end{document}